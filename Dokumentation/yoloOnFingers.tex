\documentclass[a4paper,12pt]{article}

%Dies löst einige Probleme mit der Deutschen Schreibweise
\setlength{\parindent}{0pt}
\setlength{\parskip}{5pt}
%löst Probleme mit ä,ö,ü
\usepackage[utf8]{inputenc}
%\usepackage[T1]{fontenc}
%Deutsche Silbentrennung
\usepackage[ngerman]{babel}
%Für Zitate:
\usepackage{cite}
\usepackage{url}
% Festlegung Art der Zitierung
\bibliographystyle{plain}

% Grafikpaket laden
\usepackage{graphicx}

%Farbpacket laden
\usepackage{xcolor}

%Matheformeln
\usepackage{amsmath}
\usepackage{amssymb}
\usepackage{amstext}
\usepackage{amsfonts}
\usepackage{mathrsfs}
\usepackage{dsfont}

%Todos einfach bezeichnen
\usepackage[colorinlistoftodos,prependcaption,textsize=tiny]{todonotes}
 
%Macht es möglich eine Aufzählung mit eigenen Labels zu machen
\usepackage{blindtext}
\usepackage{scrextend}
\addtokomafont{labelinglabel}{\sffamily}

%macht es möglich Tabellen zu erzeugen:
\usepackage{tabularx}
\usepackage{multirow}

%Lässt es zu, dass Gleichungen Ebenfalls beschrieben werden können.
\usepackage{float}
\usepackage{aliascnt}
\newaliascnt{eqfloat}{equation}
\newfloat{eqfloat}{h}{eqflts}
\floatname{eqfloat}{Gleichung}

\newcommand*{\ORGeqfloat}{}
\let\ORGeqfloat\eqfloat
\def\eqfloat{%
  \let\ORIGINALcaption\caption
  \def\caption{%
    \addtocounter{equation}{-1}%
    \ORIGINALcaption
  }%
  \ORGeqfloat
}


\pagestyle{headings}



\begin{document}





\thispagestyle{empty}
\begin{center}
\Large{HSR Hochschule für Technik Rapperswil}\\
\end{center}

\begin{center}
\Large{MRU: Sensors, Actors and Communication}
\end{center}
\begin{verbatim}







\end{verbatim}
\begin{center}
\textbf{\LARGE{Studienarbeit}}
\end{center}
\begin{verbatim}


\end{verbatim}
\begin{center}
\textbf{\Huge{Yolo auf Finger}}
\end{center}
\begin{verbatim}



\end{verbatim}
\begin{center}
\textbf{im Studiengang Industrial Technologies}
\end{center}
\begin{verbatim}







\end{verbatim}

\begin{flushleft}
\begin{tabular}{lll}
\textbf{eingereicht von:} & & Heinz Hofmann \flq{}hhofmann@hsr.ch\frq{}\\
& & \\
& & \\
\textbf{eingereicht am:} & & 9. Februar 2017\\
& & \\
& & \\
\textbf{Betreuer/Betreuerin:} & & Herr Prof. Dr. G. Schuster \\
& & Frau T. Mendez
\end{tabular}
\end{flushleft}









 

\todo[inline,size=\Large]{Dies ist ein Beispiel für ein Todo, und sollte vor Abgabe gelöscht werden ;)}

\newpage
%Inhaltsverzeichnis
\tableofcontents
% das Abbildungsverzeichnis
\listoffigures

\newpage
%Kapitelüberschrift
\section{Abstract}
\subsection{Aufgabenstellung}
Das Ziel dieser Projektarbeit war es, herauszufinden, ob Yolo geeignet wäre, die Fingerspitzen einer Hand in einem Bild zu klassifizieren und genau zu detektieren. 
Die Vorgaben, was die Genauigkeit betreffen lagen bei 0.1mm.
Yolo ist eine Möglichkeit, um mittels Deep-Learning Objekte in einem Bild zu klassifizieren und gleichzeitig deren genaue Position zu detektieren. 
Daher auch der Ausdruck Yolo (You only look once).
Yolo wurde als Konzept gewählt, weil es in diesem Bereich dem aktuellen Stand der Technik entspricht. 
Gerade die Geschwindigkeit dieses Netzwerks wurde als extrem hoch angepriesen (bis zu 45fps).
Diese Geschwindigkeit ist für die letztendliche Anwendung von hoher Wichtigkeit, weil es sich schlussendlich um eine Echtzeitanwendung handeln soll. 

\subsection{Vorgehen}
Mithilfe der Apparatur und Software von Tabea Méndez wurden zuerst Daten generiert. 
Um diesen Aufwand klein zu halten, wurden nur Daten vom rechten Zeigefinger generiert. 
Gleichzeitig wurde in Tensorflow die Architektur von Yolo nach gebaut. 
Dies wäre nur begrenzt nötig gewesen, da fertige Architekturen in Keras oder Darknet online zur Verfügung stehen würden. 
Um aber einen Lerneffekt im erstellen von Neuronalen Netzwerken zu erzielen, wurde trotzdem alles von Grund auf selber aufgebaut. 
Rund um die Kernarchitektur von Yolo wurde das Datenhandling, die Kostenfunktionen aber auch sämtliche Validierungen und Tests zweimal erstellt.  
Einmal für das Pretraining der Kerngewichte auf dem ImageNet Klassifizierungsdatenset und einmal für das "echte" Training auf den selber generierten Daten. 
Sobald dies alles aufgebaut und lauffähig war, wurde noch so viel wie möglich experimentiert, um herauszufinden, mit welchen Änderungen und Einstellungen das Lernresultat noch optimiert werden könnte.    

\subsection{Fazit}

Das Pretraining und auch das Training hatten seine Tücken, weil das originale Yolo-Netzwerk extrem gross ist, und entsprechend nahezu den ganzen RAM-Speicher einer GPU benötigte, wodurch nur noch begrenzt Platz für Daten übrigblieb. 
Diese Probleme konnten einigermassen umgangen werden, hatten jedoch zur Folge, dass die Bilder von 1280x960 auf 448x448 verkleinert werden mussten, um das Netzwerk zum laufen zu bringen. 
Dies hatte zur Folge, dass ein Pixel bereits bis zu 1,5mm entsprechen konnte. (Dies sollte Yolo theoretisch nicht daran hindern genauere Aussagen über die Position des Fingerspitzen zu machen.) 
Trotzdem wurde mit rund 84\% der Predictions nur eine Genauigkeit von 15mm erreicht, was in etwa 10 Pixeln entsprach. 
Mit diesem Ergebnis wurde zwar das Ziel der Aufgabenstellung (0.1mm) um Faktor 150 verpasst, allerdings in 84\% der Fälle Predictions gemacht, welche aus subjektiver menschlicher Sicht "gut" sind. 
Dies ist ein einigermaßen erstaunliches Resultat, wenn man bedenkt, dass man zum Trainieren nur rund 18'000 Bilder verwendet hatte. 
Es ist anzunehmen, dass mit einer Verbesserung der Datengewinnung und entsprechend viel mehr Daten in naher Zukunft mit diesem oder einem ähnlichen Konzept eine Genauigkeit von bis zu 1mm erreicht werden können sollte.


\newpage
\section{Daten-Pipeline}
\subsection{Bilder aufnehmen}
Die Aufnahme der Bilder geschah unverändert mit Apparatur und C++ Code von Tabea Méndez welche aus Ihrer Masterarbeit \cite{TabeasFingertracking} entstand. 
Das Ergebnis waren jeweils 8 Bilder aus einer Situation. 
Eine Situation bestand aus 4 Kameras, wobei jede Kamera jeweils ein schwarzweiß-Bild mit UV-Beleuchtung und ein schwarzweiß-Bild mit normaler weisser Beleuchtung gemacht hatte.
Wegen schlechter Erfahrungen mit Restlicht wurde der Aufbau mit schwarzem Papier abgedeckt. Diese schlechten Erfahrungen wurden gemacht, weil zu diesem Zeitpunkt zum Labeling der Daten noch keine Zeitinformation verwendet, also die Finger noch nicht von Bild zu Bild getrackt wurden.  
Pro Durchgang konnten maximal 6000 Situationen aufgenommen werden, bevor der Arbeitsspeicher des dafür verwendeten Computers an seine Grenzen kam. 

Eine Verbesserung könnte hier erreicht werden, wenn man das Programm in 2 verschiedene Threads aufteilen würde.
Dabei wäre ein Thread für das Aufnehmen der Daten und der andere für das abspeichern derselben zuständig.
So könnte \grqq{}zeitlich unbegrenzt\grqq{} Daten aufgenommen werden. Dies würde aber nur nötig, falls tatsächlich in Zukunft mit einem Roboter Daten aufgenommen würden.
\subsection{Fingerdetektion}
\begin{figure}
	\centering
	\begin{minipage}[b]{0.48\textwidth}	
		\includegraphics[trim = 270mm 90mm 90mm 170mm, clip, width=\textwidth]{Kapitel/DatenPipeline/Bilder/schlechteBoundingboxen/pic775.png}
	\end{minipage}
	\hfill
	\begin{minipage}[b]{0.48\textwidth}		
		\includegraphics[trim = 270mm 90mm 90mm 170mm, clip,width=\textwidth]{Kapitel/DatenPipeline/Bilder/schlechteBoundingboxen/pic777.png}
	\end{minipage}
	\caption{Resultate Hough-Transformation}
	\label{img:HoughTransformation}
\end{figure}
Zur Fingerdetektion wurde der Matlab-Fingerdetektor aus der Masterarbeit von Tabea Méndez \cite{TabeasFingertracking} verwendet. 
Um Zeit bei der Datenaufnahme einzusparen wurde Zeit und Rauminformation nicht miteinbezogen. 
Dies brachte einige neue Probleme mit sich.   
So wurden auch mit Restlicht beleuchtete Punkte im Hintergrund oder LED's, als Finger erkannt.
Dieses Problem konnte aber weitgehend behoben werden, indem in den Matlab-Fingerdetektor noch einige Filter eingebaut wurden.
\begin{enumerate}
\item Überspringen von Bildern, welche eine gewisse Helligkeit überschreiten. 
Dies sortiert Bilder aus, welche eine grosse Hintergrundhelligkeit und dadurch auch viele fehlerhaft erkannten Fingerspitzen enthält aus.  
\item Aussortieren von erkannten Punkten, die zu gross sind, als dass Sie ein Fingerspitz sein könnten. 
Dieser Punkt ist teilweise redundant mit dem ersten Punkt, da so grosse Punkte nur im Hintergrund bei einer extrem grossen Helligkeit auftreten können.  
\item Aussortieren von erkannten Punkten, welche zu klein sind, als dass Sie eine Fingerspitze sein könnten. 
Damit werden die meisten LED-Punkte entfernt.
\item Von den übrigen Punkten wird dann nur noch der Grösste behalten.
Diesem wird somit das Label \grqq{}rechter Zeigefingerspitz\grqq{} verliehen. 
\end{enumerate}
Mit diesen Filtern konnte ein hoher Prozentsatz der rechten Zeigefinger korrekt detektiert werden.
Auf den Finger selber bezogen war die Genauigkeit leider jedoch relativ schlecht.
Dies aus dem einfachen Grund, dass die Detektionspunkte nicht immer genau in der Mitte des Fingers zu liegen kamen. 
Weiter waren auch die Boundingboxen, welche sich aus dem Resultat der Hough-Transformation \cite{TabeasFingertracking} berechnen liessen nicht sehr genau. 
Wie man in der Abbildung \ref{img:HoughTransformation} sehen kann, können sich diese innerhalb des Fingers auch bei sehr ähnlichen Bildern stark unterscheiden.

\subsection{CSV generieren}
Das Fingerspitzentracking wurde mit Matlab gemacht und die entsprechenden Labels als .mat-File abgespeichert.
Das Deeplearning hingegen wurde mit Tensorflow und entsprechend mit Python angegangen. 
Leider war es nicht möglich mit Python direkt .mat-Files zu öffnen. 
Aus diesem Grund wurde ein kleines Matlab-Skript erstellt, welches die Labels als CSV abspeichert. 
Im Zuge dieses Skripts wurden ausserdem die Daten in Test und in Trainingsdaten aufgeteilt und je einem seperaten CSV abgespeichert.
In diesem CSV gehört jedem Bild eine Zeile. 
Pro Zeile bzw. Bild werden folgende Punkte beschrieben:
\begin{enumerate}
\item Eindeutiger Bildname, mit welchem das Bild aus dem Directory geladen werden kann. 
\item X-Koordinaten im Range [0:1280]
\item Y-Koordinaten im Range [0:960]
\item Durchmesser des Resultats der Hough-Transformation
\item Wahrscheinlichkeit, dass ein rechter Zeigefinger in diesem Bild ist. 
(Entweder 1 oder 0, je nach dem, ob ein Finger erkannt wurde.)
\end{enumerate}


\subsection{Python-Objekt generieren}
Um die Daten einfach im Trainingsprozess aufrufen zu können, wurde eigens eine kleine Python-Klasse geschrieben.
Die Daten müssen allerdings noch vor deren Verwendung im Training durch eine Funktion dieser Klasse vorverarbeitet werden.
Die Gründe für die Vorverarbeitung sind: 
\begin{enumerate}
\item Die Bilder wurden bisher Kameraweise bearbeitet.
Dies bedeutet, die Bilder heissen bei verschiedenen Kameras genau gleich.
Mit der Vorverarbeitung werden alle Bilder an einem gemeinsamen Ort gespeichert.
Ausserdem erhält jedes Bild einen neuen Namen / eine neue Nummerierung, was sie eineindeutig bezeichnen lässt.
\item Um im Training einfach mit den Label-Daten umgehen zu können und um Rechenaufwand während dem Training zu sparen wurden in der Vorverarbeitung die Labels zu demjenigen Tensor zusammengefügt, welcher in Abbildung \ref{img:label_tensor} zu sehen ist.
\item Die Distanzen X und Y sowie die Höhe und Breite der Boundingbox mussten noch normalisiert werden, damit beim Training einfacher gerechnet werden kann.
\end{enumerate}
\subsubsection{Label-Tensor}
Die Labels pro Bild sind in einem Tensor angeordnet. (Siehe Abbildung \ref{img:label_tensor})
Diese Anordnung wurde stark am Output-Tensor wie er im Yolo-Paper \cite{yolo} erscheint angelehnt.
Dabei wird das Bild in ein 7x7 Raster aufgeteilt.
Für jedes Element dieses Gitternetzes....\todo[inline]{Hier weiterschreiben}
\begin{enumerate}
\item x = Die Distanz des Zentrums der Fingerspitze zum linken Rand der Gitterzelle. Diese Distanz ist normiert auf den Bereich [0:1]. Ist der Zeigefinger nicht in dieser Gitterzelle, ist die Variable x = 0.
\item asdfasd
\end{enumerate}
%Label-Tensor
\begin{figure}	
	\centering
	\includegraphics[width=.8\textwidth]{Kapitel/DatenPipeline/Bilder/LabelTensor.pdf}
	\caption{Label-Tensor}
	\label{img:label_tensor}
\end{figure} 
\subsubsection{List of Label-Tensors}

\subsection{Daten in Neuronales Netzwerk einlesen}



\newpage
%Kapitelüberschrift
\section{Architektur} 

\subsection{Auswahl der Architektur}
Die Architektur wurde stark dem Yolo-Paper \cite{yolo} angelehnt. 
Dies obwohl zu diesem Zeitpunkt auch schon das Yolo v2-Paper \cite{yolo2} erschienen war.
Es gab damals schon viele gute Gründe dafür von Beginn weg das Netzwerk und Kostenfunktionen nach dem Yolo v2-Paper aufzubauen.
So ist Yolo v2 nach dessen Paper zu urteilen schneller und genauer. 
Der Grund, warum trotzdem Yolo v1 verwendet wurde, war dass die Beschreibung z.B. von Kostenfunktion und von der Architektur im Paper von Yolo v1 um einiges genauer und verständlicher war als im Paper von Yolo v2.
Ausserdem ging man mit der Einstellung an die Arbeit, dass wenn erst das \grqq{}einfache\grqq{} Yolo v1  erfolgreich implementiert wurde, dieses entsprechend immer noch zur 2. Version erweitert werden könnte.

\todo[inline,size=\Large]{evtl. den folgenden Abschnitt entfernen.}

Dazu kam es allerdings nicht, weil verschiedene Faktoren die erfolgreiche Fertigstellung von v1 verzögerten. 
(Genauere Informationen dazu werden später im Kapitel \todo[inline,size=\Large]{min. hier eine Referenz auf die Erklärung warum die Verzögerungen auftraten machen.} erläutert.) 
Ausserdem war das Grundlegende Ziel immer einen Eindruck dafür zu bekommen, was mit State-Of-The-Art Lösungen aktuell in diesem Bereich überhaupt möglich wäre.
Entsprechend hatte man sich auch primär auf dieses Ziel fokussiert.

\subsection{Architekturaufbau}
\label{chapter:Architekturaubau}

\begin{table}
\centering
\begin{tabularx}{1.1\textwidth}{|l|l|l|l|l|X|}
\hline
\textbf{Layer} & \textbf{Filtertyp}  & \textbf{Anzahl} & \textbf{Grösse} & \textbf{Strides} & \textbf{Output} \\
\hline 	0	& Input				&		&		&		& 448x448x1\\
\hline 	1	& Convolutional		& 64		& 7x7	& 2x2	& 224x224x64	\\
\hline 	2	& Maxpool      		& 		& 2x2	& 2x2	& 112x112x64	\\
\hline 	3   & Convolutional		& 192	& 3x3	& 1x1	& 112x112x192\\
\hline 	4	& Maxpool			& 		& 2x2	& 2x2	& 56x56x192	\\
\hline 	5	& Convolutional		& 128	& 1x1	& 1x1	& 56x56x128	\\
\hline 	6	& Convolutional		& 256	& 3x3	& 1x1	& 56x56x256	\\
\hline 	7	& Convolutional		& 256	& 1x1	& 1x1	& 56x56x256	\\
\hline 	8	& Convolutional		& 512	& 3x3	& 1x1	& 56x56x512	\\
\hline 	9	& Maxpool			&		& 2x2	& 2x2	& 28x28x512	\\
\hline 	10	& Convolutional		& 256	& 1x1	& 1x1	& 28x28x256	\\
\hline 	11	& Convolutional		& 512	& 3x3	& 1x1	& 28x28x512	\\
\hline 	12	& Convolutional		& 256	& 1x1	& 1x1	& 28x28x256	\\
\hline 	13	& Convolutional		& 512	& 3x3	& 1x1	& 28x28x512	\\
\hline 	14	& Convolutional		& 256	& 1x1	& 1x1	& 28x28x256	\\
\hline 	15	& Convolutional		& 512	& 3x3	& 1x1	& 28x28x512	\\
\hline  	16	& Convolutional		& 256	& 1x1	& 1x1	& 28x28x256	\\
\hline  	17	& Convolutional		& 512	& 3x3	& 1x1	& 28x28x512	\\
\hline 	18	& Convolutional		& 512	& 1x1	& 1x1	& 28x28x512	\\
\hline  	19	& Convolutional		& 1024	& 3x3	& 1x1	& 28x28x1024	\\
\hline  	20	& Maxpool			&		& 2x2	& 2x2	& 14x14x1024	\\
\hline  	21	& Convolutional		& 512	& 1x1	& 1x1	& 14x14x512	\\
\hline  	22	& Convolutional		& 1024	& 3x3	& 1x1	& 14x14x1024	\\
\hline  	23	& Convolutional		& 512	& 1x1	& 1x1	& 14x14x512	\\
\hline  	24	& Convolutional		& 1024	& 3x3	& 1x1	& 14x14x1024	\\
\hline  	25	& Convolutional		& 1024	& 3x3	& 1x1	& 14x14x1024	\\
\hline  	26	& Convolutional		& 1024	& 3x3	& 2x2	& 7x7x1024	\\
\hline  	27	& Convolutional		& 1024	& 3x3	& 1x1	& 7x7x1024	\\
\hline  	28	& Convolutional		& 1024	& 3x3	& 1x1	& 7x7x1024	\\
\hline 	31	& Fully-Connected	&		&(7x7x1024)x4096	&	& 4096  \\
\hline  	32	& Fully-Connected	&		& 4096x(7x7x6)	&	& 7x7x6 \\
\hline 	
\end{tabularx}
\caption{Yolo-Architektur}
\label{tbl:yolo_architektur}
\end{table}


Der grundlegende Aufbau der Architektur, wie er letztendlich aussah kann man in der Tabelle \ref{tbl:yolo_architektur} betrachten. 
Dies sah allerdings noch nicht immer so aus.
Obwohl die Convolution-Filter schon immer so ausgesehen hatten, sah der ursprüngliche Bildinput und entsprechend die Outputs der verschiedenen Layers mal anders aus.
Nach ausführlicher Diskussion \cite{PrivateCommunication} wurde zu Beginn des Architekturdesigns entschieden, dass man nicht mit 448x448 Bildern arbeitet, wie dies im Yolo-Paper \cite{yolo} gemacht wurde.
Der Grund dafür war, dass für das Training wie auch später für den Praxiseinsatz immer 1280x960 grosse Bilder zur Verfügung standen und man entsprechend nicht Informationen \grqq{}wegwerfen\grqq{} sondern so lange wie möglich im Netz behalten wollte.
So war der Input (Layer 0 in Tabelle \ref{tbl:yolo_architektur}) damals 1280x960x1, entsprechend dann auch der Output von Layer 1 nicht mehr 224x224x64 sondern 640x480x64 usw.
Da dies am Schluss nicht aufgeht mit dem Netzwerk, hatte es damals noch zwei zusätzliche Layer (29 \& 30), welche jetzt nicht mehr vorhanden sind.
Layer 29 war dabei für ein Zeropadding und Layer 30 für ein Maxpooling mit Strides 3x3 zuständig.


\begin{figure}
	\centering
	\begin{minipage}[b]{0.48\textwidth}	
		\includegraphics[width=\textwidth]{Kapitel/40Architektur/Bilder/OverflowInPretraining.pdf}
	\end{minipage}
	\hfill
	\begin{minipage}[b]{0.48\textwidth}		
		\includegraphics[width=\textwidth]{Kapitel/40Architektur/Bilder/OverflowInTraining.pdf}
	\end{minipage}
			\caption{Effekte (Extremer Anstieg der Kosten innert einer Epoche) im Pretraining(links) und im Training(rechts). x-Achse=Zeit, y-Achse=Kosten. Weinrot=Pretraining-Trainingsdaten, Hellblau=Pretraining-Validierungsdaten, Grün=Training-Trainingsdaten, Grau=Training-Validierungsdaten}
	\label{img:Overflow}
\end{figure}



Diese damalige Architektur war zu gross, um sie auch nur mit Minibatchsize=1 und Float32 ins GPU-RAM und entsprechend überhaupt zum laufen zu kriegen. 
Entsprechend wurden alle Gewichte und Knoten mit Float16 initialisiert. 
Seit dieser Initialisierung war eine Minibatchsize=7 möglich.

Mit dieser Architektur wurde eine Zeit lang trainiert, bis sich immer mehr spezielle Effekte im Training, wie auch im Pretraining (Details zum Pretraining im Kapitel \ref{chapter:Pretraining}) häuften.
Es konnte auch nach längerer Analyse nicht abschliessend geklärt werden, was die Ursache für diese Effekte (Siehe Abbildung \ref{img:Overflow}) war.
Die Vermutung lag jedoch darin, dass es sich um Overflow-Probleme im Zusammenhang mit den verwendeten Float16 handeln könnte.
Entsprechend wurde die Architektur umgebaut, sodass die Input-Bilder künstlich verkleinert wurden, um dafür Float32 verwenden zu können.
In diesem Schritt war es naheliegend, dass man sich gleich den originalen Werten, wie sie von Yolo \cite{yolo} verwendet wurden annäherte.
Entsprechend wurden die Input-Bilder auf 448x448 verkleinert.
Dies hatte wiederum zur Folge, dass seit diesem Zeitpunkt sogar eine Minibatchsize=24 verwendet werden konnte.
Ausserdem, und dies war noch viel wichtiger, traten die genannten speziellen Effekte (Abbildung \ref{img:Overflow}) weder im Pretraining noch im Training je wieder auf. 

\begin{table}
\centering
\begin{tabularx}{\textwidth}{|X|l|l|l|}
\hline  Beschreibung & Anzahl & in Bytes(Float16) & in Bytes(Float32)\\
\hline  Gewichte											& 206 M 	& 413 MB 	& 827 MB		\\
\hline  Knoten:Input=1280x960 							&  98 M 	& 186 MB 	&			\\
\hline  Knoten:Input=1280x960 Minibatchsize=7, GPU-RAM voll	& 689 M 	& 1.38 GB 	& 			\\
\hline  Knoten:Input=448x448 							&  16 M 	& 			& 64M 		\\
\hline  Knoten:Input=448x448, Minibatchsize=24, GPU-RAM voll	& 384 M 	& 			& 1.54GB 	\\
\hline
\end{tabularx}
\caption{Anzahl Gewichte und Knoten}
\label{tbl:anzahl_gewichte_knoten}
\end{table} 

Aus dieser Erfahrung kann man ableiten, dass bei Convolutional-Neural-Networks die Grösse der Input-Bilder mehr ins Gewicht fallen als die Anzahl Gewichte. 
Dies scheint nachträglich auch logisch, denn die Bilder werden während dem \grqq{}flow\grqq{} durch das Netzwerk mehrmals zwischengespeichert.
In der Tabelle \ref{tbl:anzahl_gewichte_knoten} ist die Berechnung, der Anzahl Gewichte und der Anzahl Knoten unter der Annahme, dass jedes Bild zwischen den Layern einmal als Knoten abgespeichert wird.
Dabei wurde nicht berücksichtigt, dass es pro Layer mehrere Einheiten von Knoten geben kann, wie z.B. vor und nach der Aktivierungsfunktion. 
Auch ohne diese zusätzlichen Layer kann man aber deutlich erkennen, dass wenn man die Minibatchsize solange erhöht, bis das GPU-RAM (im Rahmen dieser Arbeit 16GB) voll ist, man klar mehr Speicher für Knoten benötigt, als für Gewichte.

\subsection{Fazit}
Was die Architektur angeht kann man aus dieser Arbeit die folgenden beiden Punkte lernen:
\begin{enumerate}
\item Wenn in Convolutional Neural Networks Speicherknappheit ein Problem ist, sollte entweder die Tiefe des Netzwerks verkleinert (weniger Layer = weniger Knoten) oder der Input verkleinert (= ebenfalls weniger Knoten) werden. Nicht aber sollte man auf Bastellösungen ausweichen, sodass man sich was Datentypen angeht ausserhalb des Tensorflow-Standardbereichs aufhält. Es sei denn natürlich, man weiss ganz genau was man tut, und kennt entsprechend den Source-Code von Tensorflow in- und auswendig. Wenn dem aber so wäre, würden Sie diese Arbeit wohl kaum lesen ;) .
\item Wenn man eine Architektur nach entsprechend einer Vorlage aufbaut, sollte man nicht schon bevor man eine erfolgreich lauffähige Version hat an Parametern wie der Input-Grösse herumoptimieren. Optimieren sollte man erst, wenn man eine Lauffähige Fehlerfreie Version hat, sodass man jederzeit wieder zu dieser Lauffähigen Version zurückkehren kann. 
\end{enumerate}






\newpage
%Kapitelüberschrift
\section{Kostenfunktion} 


\newpage
%Kapitelüberschrift
\section{Tests} 
\label{chapter:tests}
\subsection{erste Ansätze}
Am Anfang ging man von der falschen Vorstellung aus, dass die Detektion der Finger über die Variable $\hat{p}_i$ im Output des Neuronalen Netzwerks (Tabelle \ref{tbl:beschr_kostenfuntion}) läuft. 
Aus diesem Grund wurden verschiedene Tests aufgebaut.

Ein Test ermittelte aufgrund von $\hat{p}_i$ und einem beliebigen Threshold einen Status für jede Gitterzelle. Diese Stati waren: 

\begin{itemize}
\item \grqq{}True-Positive\grqq{}
\item \grqq{}True-Negative\grqq{}
\item \grqq{}False-Positive\grqq{}
\item \grqq{}False-Negative\grqq{}
\end{itemize}

Wer das Kapitel \ref{chapter:design_kostenfunktion} gelesen hat, kann jetzt schon schlussfolgern, dass dies keine brauchbaren Resultate liefern konnte.
Denn wenn die Variable $\hat{p}_i$ mit andauerndem Lernen gegen 1 tendiert und nie nach unten korrigiert wird, übersteigt Sie somit irgendwann jeglichen Threshold und sagt in jeder Gitterzelle einen rechten Zeigefingerspitz voraus. 
Entsprechend ergaben die \grqq{}True-Positives\grqq{} und die \grqq{}False-Positives\grqq{} zusammen irgendwann = 1, während die \grqq{}True-Negatives\grqq{} und die \grqq{}False-Negatives\grqq{} zusammen = 0 ergaben.
Somit war dieser Test gegenstandslos und man wusste, dass die Variable $\hat{p}_i$ keine Rolle spielen wird, solange nur Labels mit einem Finger verwendet werden und entsprechend nur eine Klasse existiert.

Ein weiterer Test hatte das Ziel, dass jeweils rund um das jeweilige Label ein Kreis mit einem bestimmten Radius gelegt wird.
Die Vorhersagen wurden nun aufgeteilt in Vorhersagen, welche innerhalb des Kreises lagen, und Vorhersagen ausserhalb des Kreises.
Die Frage war nur noch, wie bestimmt man, welche der 7x7 Boundingboxen als \textbf{die} Vorhersage verwendet wurde, welche mit dem Label verglichen werden konnte.
Die Antwort ist dank dem Wissen über die Aufgabe, welche das Netz erfüllen muss relativ schnell beantwortet.
Denn wir wissen, dass in der Aufgabe, welche gelöst werden soll immer nur eine rechte Zeigefingerspitze pro Bild vorhanden sein wird.
So musste dafür kein Threshold bestimmt werden, sondern es wurde einfach diejenige Boundingbox gewählt, welche die grösste Confidence lieferte.

Mit diesem Ansatz hatte man nun ein Test, der tatsächlich etwas über das Resultat aussagte. 
So wurden verschieden grosse Kreise um die Labels gezogen um prozentuale Aussagen zu deren Genauigkeit zu kriegen.
Allerdings wurde relativ bald klar, dass es keinen grossen Sinn machen würde für jede Genauigkeit einen Kreis zu ziehen und diese dann einzeln auszuwerten.

Ausserdem wurde von Guido Schuster \cite{PrivateCommunication} in einem Gespräch folgende Bemerkung gemacht:
"Man sollte das Netzwerk darauf testen, worauf man es auch trainiert hatte."
Dieser Bemerkung folgte schliesslich die Schlussfolgerung, dass mit den Tests auch die IOU der Predictions gegenüber den Labels genauer betrachtet werden sollte.

\subsection{Letztendlicher Test}
\label{chapter:letztendlicher_test}
Aus den Erkenntnissen der ersten Ansätze konnte ermittelt werden, dass der Optimale Output aus den Tests ein Histogram, bzw. eine Wahrscheinlichkeitsdichteverteilung sein sollte.
So konnte der Code relativ schnell so angepasst werden, dass bei einem Testlauf die Distanz von Label zu Prediction (L2-Norm) für jedes Testbild in ein Element eines Vektors gespeichert wurde.
Aus diesem Vektor konnte dann ein schönes Histogramm erstellt werden, aus welchem mit einem Blick gelesen werden konnte, wie sich die Distanzen von Predictions zu den Labels über das gesamte Testset verhielten.

Nach Fertigstellung dieses Tests war es ein Leichtes dasselbe für die IOU anstelle der Distanz zu machen. 

Die Wahl der besten Prediction wurde ebenfalls nochmals verbessert.
Im Nachherein ist es ein wenig peinlich, dass in diesem Punkt so viel herumexperimentiert und ausprobiert wurde, da in der Gleichung 1 im Yolo-Paper \cite{yolo} klar ersichtlich ist, dass für die Bestimmung der besten Prediction das Produkt aus $\hat{p}_i$ und $\hat{c}_i$ massgebend ist.
Natürlich machte dies aber auch keinen Unterschied mehr, da $\hat{p}_i$ nahezu = 1 war, war die Prediction aus $\hat{p}_i * \hat{c}_i$ dieselbe wie wenn nur $\hat{c}_i$ verwendet wurde.

Somit war das Training des Netzwerks auf die Variable $\hat{p}_i$ sowie dessen Verwendung bisher nur irreführend und hatte keinen Nutzen.
Für die Zukunft aber ist es wichtig, dass dieses Yolo auch mehrere Klassen vorhersagen kann.
Dadurch macht es Sinn, diese Variable und die damit verbundenen Berechnungen in der Implementation zu belassen. 

Die Wahrscheinlichkeitsdichtefunktion, welche schliesslich bei diesen Tests durch die L2-Distanz erzeugt wurde, hatte eine etwas spezielle Form (siehe Abbildung \ref{img:dist_dichte_improved}).
Dies sorgte Anfangs für Verwirrung. 
Allerdings konnte ein Gespräch mit Guido Schuster \cite{PrivateCommunication} schnell Klarheit bringen, da es sich \grqq{}offensichtlich\grqq{} um eine Rayleigh-Verteilung handelte. 
Eine solche Verteilung entsteht, wenn zwei Gaussverteilte Variablen über die L2-Norm miteinander verbunden werden.
Da genau dies in diesem Test geschieht, war somit dieser Punkt restlos geklärt.

\subsection{Seed}
Um während den vielen Tests endlich ganz genaue Vergleiche zu erhalten wurde im Laufe der Arbeit ein fixer Seed implementiert und an Tensorflow übergeben.
Allerdings hatte dies zwei Tücken.
Dies waren auch die Gründe, warum dieser fixe Seed wieder aufgehoben wurde.

Die erste Tücke war, dass trotz der Übergabe eines fixen Seeds an Tensorflow die Ergebnisse trotzdem nicht reproduzierbar waren.
Offensichtlich hat es in Tensorflow noch weitere zufällige Werte, welche man auch mit einem festen Seed initialisieren müsste.
Diese wurden allerdings nicht gefunden.

Die zweite Tücke war jedesmal, wenn man während dem Training zwischen dem Trainingsset und dem Validierungsset hin und her wechselte.
So wurden die Bilder für das Training wieder in der genau gleichen Reihenfolge geladen wie in der letzten Epoche.
Zuerst sah es so aus, als würde der Trainingsfehler einfach in ungeheurem Tempo gegen 0 gehen, während sich der Validierungsfehler relativ schnell von jeglich vernünftigem verabschiedete.
Allerdings wurden einzig und allein die ersten paar Bilder auswendig gelernt.

Der Vorteil an der zweiten Tücke war, dass man aus einem Versehen heraus sogleich überprüft hatte, ob das Netzwerk in der Lage ist overzufitten. $==>$ Ja ist es!

\subsection{Fazit}
Bei einem Neuronalen Netzwerk sollte sich früher Gedanken gemacht werden, wie man das Resultat möglichst praxistauglich testen kann.
Dies wurde in dieser Arbeit klar falsch gemacht.
Man hatte eine funktionierende Kostenfunktion und wollte diese nach Möglichkeit verbessern.
Allerdings ist das Ziel eines neuronalen Netzwerks nicht eine tiefe Kostenfunktion zu haben, sondern den Task wozu es verwendet wird möglichst gut zu erfüllen.


\newpage
\section{Resultate}
\label{chapter:resultate}
\subsection{Testvoraussetzungen}
Das Netzwerk wurde auf 1'200'000 Bildern des ImageNet-1000-class-Datasets vortrainiert.
Danach wurde es auf rund 13'900 Bildern aus dem Testaufbau \cite{TabeasFingertracking} trainiert. 
Der Test wiederum wurde auf rund 1'500 Bildern ebenfalls aus dem Testaufbau \cite{TabeasFingertracking} getestet. 
Diese Testbilder waren dem Algorithmus während des Lernprozesses nicht zugänglich und haben entsprechend keinen Einfluss auf den Lernprozess genommen. 
Ausserdem wurden diese Bilder so gewählt, dass sie nicht gleichzeitig mit den Trainingsbildern aufgenommen wurden. 
Dies verhinderte, dass fast identische Bilder im Training und im Test vorkommen. 

\subsection{Analyse}
Um die Genauigkeit der Predictions unseres Neuronalen Netzwerkes möglichst exakt beschreiben zu können, wurden die zwei Werte Distanz und IOU gewählt (siehe Kapitel \ref{chapter:letztendlicher_test}). 
Obwohl die beiden Werte korrelieren, sagt jeder für sich nicht die volle Wahrheit über die Genauigkeit der Vorhersagen aus. 
Die Distanz ist für die geplante Anwendung der wesentlichere Wert, weil sie Informationen über den Standort der Fingerspitze im Bild preisgibt.
Die IOU ist mit der Distanz teilweise korreliert, da die IOU neben der Distanz auch von der Grösse der Boundingboxen abhängt.
Sobald die Boundingbox der Prediction und die Boundingbox des Labels sich zu überlappen beginnen, sagt die IOU etwas über die korrekte Vorhersage von Breite und Höhe der Boundingbox aus.

\subsubsection{Distanz}

%Einschätzung der Distanz:
\begin{figure}	
	\centering
	\includegraphics[width=.7\textwidth]{Kapitel/70Resultate/Bilder/DistanzenBerechnung.pdf}
	\caption{Bedeutung der normierten Distanzwerte in der realen Welt}
	\label{img:explain_normed_distance}
\end{figure}

%Beispielbilder Distanz
\begin{figure}
	\centering
	\begin{minipage}[b]{0.48\textwidth}	
		\includegraphics[width=\textwidth]{Kapitel/70Resultate/Bilder/2distKnappGut.png}
	\end{minipage}
	\hfill
	\begin{minipage}[b]{0.48\textwidth}		
		\includegraphics[width=\textwidth]{Kapitel/70Resultate/Bilder/5distKnappGut.png}
	\end{minipage}
	\caption{Prediction knapp besser als Distanz=0.02}
	\label{img:distanz_knapp_gut}
	%Eine Leerzeile einfügen	
	\begin{verbatim}
	\end{verbatim}
	\centering
	\begin{minipage}[b]{0.48\textwidth}	
		\includegraphics[width=\textwidth]{Kapitel/70Resultate/Bilder/3distKnappSchlecht.png}
	\end{minipage}
	\hfill
	\begin{minipage}[b]{0.48\textwidth}		
		\includegraphics[width=\textwidth]{Kapitel/70Resultate/Bilder/4distKnappSchlecht.png}
	\end{minipage}
	\caption{Prediction knapp schlechter als Distanz=0.02}	
	\label{img:distanz_knapp_schlecht}
\end{figure}

%Komplette Wahrscheinlichkeits-Dichte-Funktion der Distanz
\begin{figure}	
	\centering
	\includegraphics[width=.7\textwidth]{Kapitel/70Resultate/Bilder/distProbDensity.pdf}
	\caption{Komplette Wahrscheinlichkeits-Dichte-Funktion der Distanz (Grenze: Dist=0.02)}
	\label{img:dist_dichte}
\end{figure}
%Komplette Wahrscheinlichkeits-Dichte-Funktion mit Logarithmischer y-Achse
\begin{figure}	
	\centering
	\includegraphics[width=.7\textwidth]{Kapitel/70Resultate/Bilder/logdistProbDensity.pdf}
	\caption{Komplette Wahrscheinlichkeits-Dichte-Funktion der Distanz mit Logarithmischer y-Achse (Grenze: Dist=0.02)}
	\label{img:log_dist_dichte}
\end{figure}
%Wahrscheinlichkeits-Dichtefunktion der Distanz. Ausreisser nicht miteingerechnet
\begin{figure}	
	\centering
	\includegraphics[width=.7\textwidth]{Kapitel/70Resultate/Bilder/distProbDensity_improved.pdf}
	\caption{Wahrscheinlichkeits-Dichtefunktion der Distanz. Ausreisser nicht miteingerechnet (Grenze: Dist=0.02)}
	\label{img:dist_dichte_improved}
\end{figure}

Die Distanz beschreibt die normierte Differenz zwischen dem Zentrumspunkt des Labels und dem Zentrumspunkt der Vorhersage. 
Sämtliche Distanzen wurden so normiert, dass die Höhe des Bildes und auch die Breite gleich eins sind. 
Die maximale Distanz zwischen zwei Punkten ist also die Diagonale über ein Bild, welche entsprechend sqrt(2) ist.
Was diese Normierten Distanzen in der realen Welt bedeuten ist auf Abbildung \ref{img:explain_normed_distance} erklärt. 
Zum Vergleich, ein menschlicher Zeigefinger ist zwischen 10 und 20 mm breit.
Eine normierte Distanz von 0.02 entspricht auf unserem Versuchsaufbau somit ziemlich genau der Breite eines menschlichen Fingers. 

Um die Resultate in gut und schlecht einteilen zu können wurde ein Threshold von 0.02 definiert.
Die Definition dieses Thresholds wurde gemacht, indem Bilder zusammen mit der entsprechenden Distanz analysiert wurden.
Der Wert 0.02 entspricht somit derjenigen Distanz, welche gerade noch knapp annehmbar ist, um einen Finger als detektiert gelten zu lassen.
Um ein Gefühl für diese Distanzen zu bekommen, lohnt es sich die Abbildungen \ref{img:distanz_knapp_gut} \& \ref{img:distanz_knapp_schlecht} anzusehen, welche Bilder zeigen, die eine Distanz nahe dieses Thresholds aufweisen. 

Um die Verteilung der Distanzen gut verstehen zu können, ist in Abbildung \ref{img:dist_dichte} eine Wahrscheinlichkeitsdichte der Distanzen im Testset zu sehen. Diese Dichtefunktion wurde erst nach der Bestimmung des Thresholds erzeugt und zeigt, dass rund 84\% der Distanzen kürzer sind als 0.02 und somit die entsprechenden Finger \grqq{}erfolgreich\grqq{} erkannt wurden.

Erstaunlich ist auch, dass die Distanzen, welche grösser als 0.25 sind in der Wahrscheinlichkeitsdichte in kleinen Bündeln vorkommen. 
Dies lässt darauf schliessen, dass die Trainingsdaten nicht komplett Bias-Frei sind.
Erstaunlich ist auch, dass es um die Distanz von 0.4 die Wahrscheinlichkeitsdichte als eine Art Bündel auftritt.
Dies ist in der Abbildung \ref{img:log_dist_dichte}, welche eine logarithmisches Abbild von Abbildung \ref{img:dist_dichte} ist, gut zu sehen.
Nach kurzer Kontrolle konnte tatsächlich festgestellt werden, dass z.B. bei einer Distanz von ca. 0.4 immer ein bestimmter Punkt des Hintergrundes vorhergesagt wurde, welcher sehr selten in den Labels als Finger markiert wurde. 

Die Statistik sollte nicht von Ausreissern, welche aufgrund von falschen Labels entstanden sind,verfälscht werden.
Deshalb wurde wie in Abbildung \ref{img:dist_dichte_improved} noch eine zweite Wahrscheinlichkeitsdichte-Funktion erstellt. 
Dabei wurden alle Distanzen, welche grösser als 0.25 waren, gelöscht.

Wie man ausserdem aus Kapitel \ref{chapter:fingerdetektion} entnehmen kann, gab es schon in der Erzeugung der Labels eine gewisse Unschärfe.
So hat auf Abbildung \ref{img:Erosion}  der y-Wert im Vergleich vom einen Bild zum anderen und relativ zum Finger eine Differenz von rund 10 Pixeln, was im normierten Mass einer Distanz von rund 0.02 entspricht.
Um dieses Problem aufzeigen zu können wurde ein Bild mit einem klaren Unterschied verwendet, was bedeutet dass die meisten Labels Fehler haben, die kleiner als diese Grenze sind.
Nichtsdestotrotz wird man wohl nie in der Lage sein, viel bessere Resultate in der Genauigkeit einzufahren, wenn die Labels noch solche Abweichungen aufweisen.

Es kann sehr gut sein, dass in der Arbeit \grqq{}Hand Pose Estimation\grqq{} \cite{HandPoseEstimation} bessere Labels erzeugt wurden.
Denn in dieser Arbeit\cite{HandPoseEstimation} wurden die Fingerspitzen im 3D-Raum bestimmt und anschliessend in den 2D-Raum zurückgemappt.
Ausserdem wurde eine Fehlerrechnung gemacht, wie weit die 2D-Labels mit den zurückprojezierten Labels aus dem 3D-Raum übereinstimmen. 
Allerdings wurde in der erwähnten Arbeit\cite{HandPoseEstimation} kein Vergleich gemacht wie dies in dieser Arbeit der Fall ist.
Deshalb kann keine abschliessende Aussage zu diesem Thema gemacht werden.

\subsubsection{Intersection Over Union IOU}
%Beispielbilder IOU
\begin{figure}
	\centering
	\begin{minipage}[b]{0.48\textwidth}	
		\includegraphics[width=\textwidth]{Kapitel/70Resultate/Bilder/7iouKnappGut.png}
	\end{minipage}
	\hfill
	\begin{minipage}[b]{0.48\textwidth}		
		\includegraphics[width=\textwidth]{Kapitel/70Resultate/Bilder/8iouKnappGut.png}
	\end{minipage}
	\caption{Prediction knapp besser als IOU=0.4}
	\label{img:iou_knapp_gut}
	%Eine Leerzeile einfügen	
	\begin{verbatim}
	\end{verbatim}
	\centering
	\begin{minipage}[b]{0.48\textwidth}	
		\includegraphics[width=\textwidth]{Kapitel/70Resultate/Bilder/6iouKnappSchlecht.png}
	\end{minipage}
	\hfill
	\begin{minipage}[b]{0.48\textwidth}		
		\includegraphics[width=\textwidth]{Kapitel/70Resultate/Bilder/9iouKnappSchlecht.png}
	\end{minipage}
	\caption{Prediction knapp schlechter als IOU=0.4}
	\label{img:iou_knapp_schlecht}
\end{figure}
%Wahrscheinlichkeits-Dichte-Funktion der IOU
\begin{figure}
	\centering
	\includegraphics[width=.7\textwidth]{Kapitel/70Resultate/Bilder/IOUprobDensity.pdf}
	\caption{Wahrscheinlichkeits-Dichte-Funktion der IOU (Grenze: IOU=0.4)}
	\label{img:iou_dichte}
\end{figure}


Die IOU beschreibt die Überlappung der vorhergesagten Boundingbox und der Boundingbox des Labels. 
Daher sagt die IOU etwas über die korrekte Grösse der Boundingbox, sowie deren korrekte Lage aus. 
Um wieder etwas über gut und schlecht aussagen zu können, wurde wieder ein Threshold definiert (0.4).
Da durch die IOU, wie erwähnt, mehrere Faktoren beschrieben werden, ist die Grenze verschwommener. 
So gibt es nach menschlicher Ansicht hervorragende Vorhersagen, welche eine IOU von 0.3 haben und wiederum mässige Vorhersagen mit einer IOU von nahezu 0.4.
Um ein Gefühl für diesen Threshold zu bekommen, lohnt es sich, die Abbildungen \ref{img:iou_knapp_gut} \& \ref{img:iou_knapp_schlecht} zu berücksichtigen.
So fiel die Entscheidung den Threshold konservativ zu wählen, sodass nur Werte als gut erachtet werden können, welche auch gut sind. 

Auch für die IOU gibt es zur Übersicht eine Wahrscheinlichkeitsdichte die in Abbildung \ref{img:iou_dichte} betrachtet werden kann.
Aus dieser Grafik kann gelesen werden, dass rund 6\% der Vorhersagen klar falsch sind, weil die IOU nur Null ist, wenn sich die beiden Boundingboxen nicht berühren. Entsprechend kann gesagt werden, dass rund 94\% der Vorhersagen zumindest sehr grob richtig sind, weil sich bei diesen 94\% die Boundingboxen von Label und Prediction zumindest ein ganz kleines bisschen überlappen. 

Genau wie bei der Distanz gibt es auch bei der IOU, bzw. bei den Boundingboxen eine gewisse Unschärfe in den Labels (siehe Kapitel \ref{chapter:fingerdetektion}).
So haben die beiden Bilder in Abbildung \ref{img:Erosion} wenn man Sie zueinander normiert gegenüber einander eine IOU von knapp $0.4$.
Wie bei den Labelfehlern in der Distanz dürften die Fehler auch bei den Boundingboxen klar kleiner als im hier berechneten Beispiel sein.
Nichtsdestotrotz wird man wohl auch hier nie in der Lage sein viel bessere Resultate in der Genauigkeit einzufahren, wenn die Labels noch solche Abweichungen aufweisen.

\subsection{Erkenntnis}
Dass die Fehler in Distanz und IOU nahezu genau auf die gesetzten Grenzen zur Bewertung der Resultate fielen, war reiner Zufall.
So sind die Grenzen gemacht worden, bevor IOU und Distanz der Labels dieser zwei Bilder (Abbildung \ref{img:Erosion}) zueinander berechnet wurden.

Diese Erkenntnis jedoch relativiert sämtliche Resultate, wie sie in diesem Kapitel beschrieben wurden.
Angenommen es könnten noch mit perfekten Labels gearbeitet werden, gibt es folgende Hypothesen, wie sich die Vorhersagen verhalten werden:
\begin{enumerate}
\item Die Vorhersagen werden ebenfalls besser, weil es sich bei den Fehlern in den Labels um einen Bias und nicht bloss um Varianz gehandelt hatte.
\item Die Vorhersagen bleiben gleich gut, weil es sich bei den Fehlern in den Labels um Varianz und nicht um einen Bias gehandelt hatte.
\end{enumerate}
Beide Hypothesen wären möglich, können aber leider erst überprüft werden, wenn \grqq{}perfekte\grqq{} Label-Daten zur Verfügung stehen. 





\newpage
%Kapitelüberschrift
\section{Pretraining}

 


\newpage
%Kapitelüberschrift
\section{Fazit} 


\newpage
%sicherstellen, dass Literaturverzeichnis auch im Inhaltsverzeichnis aufgeführt wird.
\addcontentsline{toc}{section}{Literatur}
%Literaturverzeichnis anzeigen:
\bibliography{Kapitel/90Literaturverzeichnis/literatur}

\end{document}
