\newpage
%Kapitelüberschrift
\section{Architektur} 

\subsection{Auswahl der Architektur}
Die Architektur wurde stark dem Yolo-Paper \cite{yolo} angelehnt. 
Dies obwohl zu diesem Zeitpunkt auch schon das Yolo v2-Paper \cite{yolo2} erschienen war.
Es gab damals schon viele gute Gründe dafür von Beginn weg das Netzwerk und Kostenfunktionen nach dem Yolo v2-Paper aufzubauen.
So ist Yolo v2 nach dessen Paper zu urteilen schneller und genauer. 
Der Grund warum dies nicht gemacht wurde war, dass Yolo v1 nicht ganz so komplex aufgebaut war, und auch noch nicht darauf abzielte möglichst viele verschiedene Klassen zu erkennen. 
(Das Ziel dieser Arbeit ist schliesslich eine begrenzte Anzahl von Klassen zu detektieren.)
Man ging mit der Einstellung an die Arbeit, dass wenn erst das \grqq{}einfache\grqq{} Yolo v1  erfolgreich implementiert wurde, dieses entsprechend immer noch zur 2. Version erweitert werden könnte.

\todo[inline,size=\Large]{evtl. den folgenden Abschnitt entfernen.}

Dazu kam es allerdings nicht, weil verschiedene Faktoren die erfolgreiche Fertigstellung von v1 verzögerten. 
(Genauere Informationen dazu werden später im Kapitel \todo[inline,size=\Large]{min. hier eine Referenz auf die Erklärung warum die Verzögerungen auftraten machen.} erläutert.) 
Ausserdem war das Grundlegende Ziel immer einen Eindruck dafür zu bekommen, was mit State-Of-The-Art Lösungen aktuell in diesem Bereich überhaupt möglich wäre.
Entsprechend hatte man sich auch primär auf dieses Ziel fokussiert.

\subsection{Architekturaufbau}
Der grundlegende Aufbau der Architektur, wie er letztendlich aussah kann man in der Tabelle \ref{tbl:yolo_architektur} betrachten. 
Dies sah allerdings noch nicht immer so aus.
Obwohl die Convolution-Filter schon immer so ausgesehen hatten, sah der ursprüngliche Bildinput und entsprechend die Outputs der verschiedenen Layers mal anders aus.
Nach ausführlicher Diskussion \cite{PrivateCommunication} wurde zu Beginn des Architekturdesigns entschieden, dass man nicht mit 448x448 Bildern arbeitet, wie dies im Yolo-Paper \cite{yolo} gemacht wurde.
Der Grund dafür war, dass für das Training wie auch später für den Praxiseinsatz immer 1280x960 grosse Bilder zur Verfügung standen und man entsprechend nicht Informationen \grqq{}wegwerfen\grqq{} sondern so lange wie möglich im Netz behalten wollte.
So war der Input (Layer 0 in Tabelle \ref{tbl:yolo_architektur}) damals 1280x960x1, entsprechend dann auch der Output von Layer 1 nicht mehr 224x224x64 sondern 640x480x64 usw.
Da dies am Schluss nicht aufgeht mit dem Netzwerk, hatte es damals noch zwei zusätzliche Layer (29 \& 30), welche jetzt nicht mehr vorhanden sind.
Layer 29 war dabei für ein Zeropadding und Layer 30 für ein Maxpooling mit Strides 3x3 zuständig.

Diese damalige Architektur war zu gross, um sie auch nur mit Batchsize=1 und Float32 zum laufen zu kriegen. 
Entsprechend wurden alle Gewichte und Knoten mit Float16 initialisiert. 
Seit dieser Initialisierung war eine Batchsize=7 möglich.

Mit dieser Architektur wurde eine Zeit lang trainiert, bis sich immer häufiger spezielle Effekte im Training, wie auch im Pretraining häuften.
Es konnte auch nach langer Analyse nicht abschliessend geklärt werden, was die Ursache für diese Effekte (Siehe Bild \todo[inline,size=\Large]{Hier noch Referenz auf Bilder  mit speziellen Effekten einfügen.}) war.
\todo[inline,size=\Large]{Hier Bilder der speziellen Effekte im Pretraining und im Training einfügen}
Die Vermutung lag jedoch darin, dass es sich um Overflow-Probleme im Zusammenhang mit den verwendeten Float16 handeln könnte.
Entsprechend wurde die Architektur umgebaut, sodass die Input-Bilder künstlich verkleinert wurden, um dafür Float32 verwenden zu können.
In diesem Schritt war es naheliegend, dass man sich gleich den originalen Werten, wie sie von Yolo \cite{yolo} verwendet wurden annäherte.
Entsprechend wurden die Input-Bilder auf 448x448 verkleinert.
Dies hatte wiederum zur Folge, dass seit diesem Zeitpunkt sogar eine Batchsize=24 verwendet werden konnte.
Ausserdem, und dies war noch viel wichtiger, traten die genannten speziellen Effekte weder im Pretraining noch im Training je wieder auf. 

Aus dieser Erfahrung kann man ableiten, dass bei Convolutional-Neural-Networks die Grösse der Input-Bilder mehr ins Gewicht fallen als die Anzahl Gewichte. 
Dies scheint nachträglich auch logisch, denn die Bilder werden während dem \grqq{}flow\grqq{} durch das Netzwerk mehrmals zwischengespeichert.
In der Tabelle \ref{tbl:anzahl_gewichte_knoten} ist die Berechnung, der Anzahl Gewichte und der Anzahl Knoten unter der Annahme, dass jedes Bild zwischen den Layern einmal als Knoten abgespeichert wird.
Dabei wurde nicht berücksichtigt, dass es pro Layer mehrere Einheiten von Nodes geben kann, wie z.B. vor und nach der Aktivierungsfunktion. 
Auch ohne diese zusätzlichen Layer kann man aber deutlich erkennen, dass wenn man die Batchsize solange erhöht, bis das GPU-RAM (im Rahmen dieser Arbeit 16GB) voll ist, man klar mehr Speicher für Knoten benötigt, als für Gewichte.

Daraus konnte folgendes gelernt werden: Wenn in Convolutional Neural Networks Speicherknappheit ein Problem ist, sollte entweder das Netzwerk verkürzt (weniger Layer = weniger Knoten) oder die Input-Bilder verkleinert (= ebenfalls weniger Knoten) werden.
Nicht aber sollte man auf Bastellösungen ausweichen, dass man sich was Datentypen angeht ausserhalb des Standardbereichs aufhält. 
Ausser natürlich, man weiss was man tut und kennt entsprechend den Source-Code von Tensorflow in- und auswendig.

\begin{table}
\centering
\begin{tabularx}{\textwidth}{|X|l|l|l|}
\hline  Beschreibung & Anzahl & in Bytes(Float16) & in Bytes(Float32)\\
\hline  Gewichte											& 206 M 	& 413 MB 	& 827 MB		\\
\hline  Knoten:Input=1280x960 							&  98 M 	& 186 MB 	&			\\
\hline  Knoten:Input=1280x960 Batchsize=7, GPU-RAM voll	& 689 M 	& 1.38 GB 	& 			\\
\hline  Knoten:Input=448x448 							&  16 M 	& 			& 64M 		\\
\hline  Knoten:Input=448x448, Batchsize=24, GPU-RAM voll	& 384 M 	& 			& 1.54GB 	\\
\hline
\end{tabularx}
\caption{Anzahl Gewichte und Knoten}
\label{tbl:anzahl_gewichte_knoten}
\end{table} 

\subsection{}

\begin{table}
\centering
\begin{tabularx}{1.1\textwidth}{|l|l|l|l|l|X|}
\hline
\textbf{Layer} & \textbf{Filtertyp}  & \textbf{Anzahl} & \textbf{Grösse} & \textbf{Strides} & \textbf{Output} \\
\hline 	0	& Input				&		&		&		& 448x448x1\\
\hline 	1	& Convolutional		& 64		& 7x7	& 2x2	& 224x224x64	\\
\hline 	2	& Maxpool      		& 		& 2x2	& 2x2	& 112x112x64	\\
\hline 	3   & Convolutional		& 192	& 3x3	& 1x1	& 112x112x192\\
\hline 	4	& Maxpool			& 		& 2x2	& 2x2	& 56x56x192	\\
\hline 	5	& Convolutional		& 128	& 1x1	& 1x1	& 56x56x128	\\
\hline 	6	& Convolutional		& 256	& 3x3	& 1x1	& 56x56x256	\\
\hline 	7	& Convolutional		& 256	& 1x1	& 1x1	& 56x56x256	\\
\hline 	8	& Convolutional		& 512	& 3x3	& 1x1	& 56x56x512	\\
\hline 	9	& Maxpool			&		& 2x2	& 2x2	& 28x28x512	\\
\hline 	10	& Convolutional		& 256	& 1x1	& 1x1	& 28x28x256	\\
\hline 	11	& Convolutional		& 512	& 3x3	& 1x1	& 28x28x512	\\
\hline 	12	& Convolutional		& 256	& 1x1	& 1x1	& 28x28x256	\\
\hline 	13	& Convolutional		& 512	& 3x3	& 1x1	& 28x28x512	\\
\hline 	14	& Convolutional		& 256	& 1x1	& 1x1	& 28x28x256	\\
\hline 	15	& Convolutional		& 512	& 3x3	& 1x1	& 28x28x512	\\
\hline  	16	& Convolutional		& 256	& 1x1	& 1x1	& 28x28x256	\\
\hline  	17	& Convolutional		& 512	& 3x3	& 1x1	& 28x28x512	\\
\hline 	18	& Convolutional		& 512	& 1x1	& 1x1	& 28x28x512	\\
\hline  	19	& Convolutional		& 1024	& 3x3	& 1x1	& 28x28x1024	\\
\hline  	20	& Maxpool			&		& 2x2	& 2x2	& 14x14x1024	\\
\hline  	21	& Convolutional		& 512	& 1x1	& 1x1	& 14x14x512	\\
\hline  	22	& Convolutional		& 1024	& 3x3	& 1x1	& 14x14x1024	\\
\hline  	23	& Convolutional		& 512	& 1x1	& 1x1	& 14x14x512	\\
\hline  	24	& Convolutional		& 1024	& 3x3	& 1x1	& 14x14x1024	\\
\hline  	25	& Convolutional		& 1024	& 3x3	& 1x1	& 14x14x1024	\\
\hline  	26	& Convolutional		& 1024	& 3x3	& 2x2	& 7x7x1024	\\
\hline  	27	& Convolutional		& 1024	& 3x3	& 1x1	& 7x7x1024	\\
\hline  	28	& Convolutional		& 1024	& 3x3	& 1x1	& 7x7x1024	\\
\hline 	31	& Fully-Connected	&		&(7x7x1024)x4096	&	& 4096  \\
\hline  	32	& Fully-Connected	&		& 4096x(7x7x6)	&	& 7x7x6 \\
\hline 	
\end{tabularx}
\caption{Yolo-Architektur}
\label{tbl:yolo_architektur}
\end{table}
