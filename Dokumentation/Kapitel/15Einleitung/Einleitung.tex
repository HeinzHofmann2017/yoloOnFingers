\newpage
\section{Einleitung}
\subsection{Ausgangslage}
An der Hochschule für Technik Rapperswil wurde ein Flugsimulator gebaut, welcher  mithilfe eines durch Motoren bewegbaren Sitzes und einer Virtual-Reality-Brille ein extrem echtes Fluggefühl vermittelt.

Um den Simulator möglichst echt und kommerziell nutzbar zu machen, muss die Person, welche die Virtual-Reality-Brille aufgesetzt hat die Knöpfe im Cockpit bedienen können. 
Dies geht selbstverständlich nur, wenn diese Person auch ihre eigenen Finger sieht.
Momentan ist dies nicht der Fall.
Und an diesem Punkt kommt \grqq{}Yolo auf Finger\grqq{} ins Spiel.

Das Ziel war es mithilfe von vier Kameras die Finger des Piloten in Echtzeit und im 3D-Raum zu tracken und anschliessend animiert in der Virtual-Reality-Brille wieder darzustellen.

Einige erste Schritte in diese Richtung wurden bereits mit der Arbeit \grqq{}Finger-spitzen-Tracking im 3D-Raum\grqq{} \cite{TabeasFingertracking} erbracht.
So wurde ein System entwickelt, welches mit vier Kameras Punkte von den 2D-Bilder in den 3D-Raum mappen kann.
Ausserdem wurde im selben Rahmen ein Testaufbau gemacht, welcher es erlaubt mithilfe von UV Licht Label-Daten aufzunehmen.

\subsection{Ziel}
Es soll Folgendes herausgefunden werden. 
Sind Algorithmen, welche auf dem Konzept von Yolo\cite{yolo} aufbauen geeignet um im 2D-Raum die Positon von Fingerspitzen auf 0.1mm genau zu finden?

\subsection{Hauptquellen}
Diese Studienarbeit wurde vor allem auf der Arbeit \grqq{}You Only Look Once\grqq{}\cite{yolo} aufgebaut.
Darin wird die Architektur eines Neuronalen Netzwerks beschrieben, welches mit im Vergleich hervorragender Performance und einer nur leicht schlechteren Genauigkeit Gegenstände und Objekte in 2D-Bildern erkennt und deren Standort im Bild detektiert.
Diese Arbeit ist zusammen mit Ihrer Nachfolgearbeit \grqq{}Yolo9000\grqq{} \cite{yolo2} der aktuelle State of the Art was Klassifizierung kombiniert mit Detektion angeht. \cite{yolo} \cite{yolo2}

Die Arbeit \grqq{}Fingerspitzen-Tracking im 3D-Raum\grqq{} von Tabea Méndez \cite{TabeasFingertracking} war gerade wegen der gebauten Vorrichtung zur Datengeneration der Ausgangspunkt für diese Arbeit.

Teilweise als theoretische Grundlage diente das Buch Deeplearning\cite{deeplearning}, welches im Inhalt ebenfalls dem aktuellen State of the Art entspricht.

Die wahrscheinlich wertvollste Quelle dieser Arbeit waren die Diskussionen und Anregungen von und mit Prof. Dr. Guido Schuster, Tabea Méndez, Hannes Badertscher und Jonas Schmid \cite{PrivateCommunication}. An dieser Stelle ein herzliches Dankeschön an die betroffenen Personen, welche geduldig und kompetent für jegliche Fragen zur Verfügung standen.

\subsection{Vorgehen}
\label{chapter:vorgehen}
In einem ersten Schritt wurde mit Hilfe des Testaufbaus\cite{TabeasFingertracking} genügend Daten aufgenommen, damit ein Neuronales Netzwerk darauf trainiert werden kann. (Kapitel \ref{chapter:daten_pipeline})

Als nächstes wurde Programmierung von Tensorflow in Python erarbeitet und gelernt.

Danach wurde eine erste, möglichst einfache Version von Yolo v1\cite{yolo} implementiert. (Kapitel \ref{chapter:architektur} \& \ref{chapter:kostenfunktion})

Sobald diese Version lauffähig war, wurde diese optimiert, sodass damit ein möglichst gutes Resultat erzeugt werden konnte.

Gleichzeitig wurde ein Test entwickelt, welcher letztendlich zeigen soll, wie es um die Qualität der Vorhersagen steht. (Kapitel \ref{chapter:tests}) 

Als letztes sollte das Netzwerk noch auf Yolo v2\cite{yolo2} erweitert werden. 
Dazu kam es aber aus Gründen der Zeit und der Planung nicht mehr.

\subsection{Aufbau}
In den Kapiteln zwei bis fünf werden die einzelnen Punkte im Aufbau des Netzwerks vom Erzeugen der Daten bis zum Test der Resultate genau ausgeleuchtet.
In diesen Kapiteln wird auch beschrieben, welche Hürden oder Schwierigkeiten sich zeigten und wie diese falls möglich überwunden wurden.
Im Kapitel sechs, werden die erreichten Resultate dieser Arbeit beschrieben.
In diesem Kapitel wird ersichtlich, wo diese Technologie aktuell steht, und was damit alles möglich ist.

Bevor im letzten Kapitel ein Fazit zur gesamten Arbeit gezogen wird, wird im Kapitel sieben genauer auf die Art und Weise eingegangen, wie das Netzwerk vor-trainiert (Pretraining) wurde. 
Dieser Punkt erhält ein eigenes Kapitel, weil diese Arbeit \grqq{}parallel\grqq{} zum eigentlichen Training aufgebaut und ausgeführt wurde.

Das Ziel dieser Dokumentation ist es schnell einen Überblick über Gelerntes und die aufgetretenen Probleme erhalten.
Dies soll künftigen Studenten/Assistenten helfen, welche sich mit dem Thema auseinandersetzen.
So sollten allfällige Fehler nicht zweimal gemacht werden.

